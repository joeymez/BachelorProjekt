% !TEX root = Hauptteil.tex

% LÄNGE: ca. 12 Seiten
\chapter{Teil II: Agile Arbeitsweisen in der Praxis}
\minitoc 
\vspace{1 cm} 

%------------------------------------------------------
% KAPITEL 3.1 - Projektmanagement im Konzern
%------------------------------------------------------
\section{Gegen"uberstellung von agilen zu klassischen Modellen}
Inhalt ...
% - Kurze Vorstellung

\subsection{Kundenverf"ugbarkeit}
Inhalt ...


\subsection{Scope \& Feuatures}
Inhalt ...

\subsection{Feature Priorisierung}
Inhalt ...

\subsection{Teams \& F"uhrung}
Inhalt ...

\subsection{Finanzierung}
Inhalt ...


%------------------------------------------------------
% KAPITEL 3.2 - PROJEKTMANAGEMENT IN DER INDUSTRIE
%------------------------------------------------------
\section{Agile Crossfunktionale Projekte}
Inhalt ...
% - agile@MAN Community
% - evtl. GD miteinbeziehen

\subsection{Agile Ans"atze}
Inhalt ...
% - SCRUM
% - KANBAN-Board (siehe TRELLO)

\subsection{Werkzeuge \& Features}
Inhalt ...
% - TRELLO (webbasierte PM-Software)
% - SLACK (Kommunikationsplattform)