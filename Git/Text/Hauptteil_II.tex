% !TEX root = Hauptteil.tex

% LÄNGE: ca. 12 Seiten
\chapter{Teil II: IT-Projektmanagement in der Praxis}
\minitoc 
\vspace{1 cm} 

%------------------------------------------------------
% KAPITEL 3.1 - Projektmanagement im Konzern
%------------------------------------------------------
\section{MAN Truck \& Bus AG}
Inhalt ...
% - Kurze Vorstellung

\subsection{IT-Projektmanagement im Konzern}
Inhalt ...
% - SAP Projekte (siehe Projekte aus Praktikum)
% - Agile Projekte (siehe RIO, etc... )

\subsection{Agile IT-Projekte bei der MAN}
Inhalt ...
% - RIO
% - Begleitung durch GDT (siehe Abt. Cross Digital Projects)


%------------------------------------------------------
% KAPITEL 3.2 - PROJEKTMANAGEMENT IN DER INDUSTRIE
%------------------------------------------------------
\section{Cross Digital Projects (GDT)}
Inhalt ...
% - Kurze Vorstellung der Abteilung

\subsection{Konzept: Agiles Vorgehen}
Inhalt ...
% - DERZEIT NOCH NICHT GENAU DEFINIERT (Im Laufe der Zeit schreiben)

\subsection{Vorgehensmodelle}
Inhalt ...
% - SCRUM
% - KANBAN-Board (siehe TRELLO)

\subsection{Management Tools}
Inhalt ...
% - TRELLO (webbasierte PM-Software)
% - SLACK (Kommunikationsplattform)