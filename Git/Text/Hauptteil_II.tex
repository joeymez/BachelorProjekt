% !TEX root = Hauptteil.tex

% LÄNGE: ca. 12 Seiten
\chapter{Teil I: Agile Arbeitsweisen in der Praxis}
\minitoc 
\vspace{1 cm} 



%------------------------------------------------------
% KAPITEL 3.1 - Agile Organisationen & ihre Bedeutung
%------------------------------------------------------
\section{Agile Organisationen \& ihre Bedeutung}
Inhalt ...
% - Kurze Vorstellung

\subsection{Kl"arungsbedarf: 'Agilit"at'}
Inhalt ...

\subsection{Notwendigkeit agiler Arbeitsweisen}
Inhalt ...

\subsection{Merkmale einer agilen Organisation}
Inhalt ...



%------------------------------------------------------
% KAPITEL 3.2 - Gegenüberstellung von agilen zu klassischen Modellen
%------------------------------------------------------
\section{Gegen"uberstellung von agilen zu klassischen Modellen}
Inhalt ...
% - Kurze Vorstellung

\subsection{Kundenverf"ugbarkeit}
Inhalt ...


\subsection{Scope \& Feuatures}
Inhalt ...

\subsection{Feature Priorisierung}
Inhalt ...

\subsection{Teams \& F"uhrung}
Inhalt ...

\subsection{Finanzierung}
Inhalt ...

\subsection{Qualit"atsmanagement}
Inhalt ...


%------------------------------------------------------
% KAPITEL 3.3 - Agile Frameworks
%------------------------------------------------------
\section{Agile Frameworks}
% Siehe PDF (AO&B)Die agile Organisation ‘in a nutshell’ - Silvester Schmidt - Schwarmorganisation
% -> Unter Agile Skalierung ... kann 1:1 so übernommen werden mit Änderungen

\subsection{Traditionelle Regelwerke}
% - Scrum
% - Kanban
% - XP (Extreme Programming)
% - FDD (Feature-Driven Development)

\subsection{Konzeptionelle Regelwerke}
% - Scrum of Scrums
% - Nexus
% - Large-Scale Scrum (LeSS)
% - Scaled Agile Frameworks (SAFe)
% - Scrum@Scale
% - DAD (Disciplined Agile Delivery) -> nicht Skallierte Projekte
% - Agile Scaling Cycle -> nicht Skallierte Projekte
% - Enterprise Scrum -> nicht Skallierte Projekte


