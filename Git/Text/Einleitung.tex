% !TEX root = Einleitung.tex


\chapter{Einleitung}
\minitoc
\vspace{1 cm} 

Die Vernetzung der Produktion in der Industrie durch die \gls{industrie_4.0}, die enorme 
Ansammlung und Auswertung von privaten und "offentlichen Daten (\gls{big_data}), sowie die 
Verkn"upfung von allt"aglichen Gegenst"anden mit dem Internet (\gls{iot}) -- dies sind nur
einige Begrifflichkeiten, die das digitale Zeitalter definieren. Der Umbruch f"ur diese 
sogenannte \textit{Digitale Revolution} ist unter Fachkreisen auf den Beginn des 21. 
Jahrhunderts zur"uckzuf"uhren \cite{qeins} (G. Braunberger). Seither hat diese neue digitale 
Welt sowohl im "offentlichen, im wirtschaftlichen, wie auch im privaten Sektor f"ur positive Bilanz 
und Begeisterung gesorgt. Nicht nur das es heutzutage beispielsweise mit Hilfe der 
\textit{\gls{digitalisierung}} m"oglich ist Informationen schneller und einfacher zu verarbeiten, 
sondern das anhand dessen ebenfalls die Automatisierung in der Wirtschaft vorangetrieben 
wird \cite{qzwei} (Prof. Dr. O. Bendel). Insbesondere durch die bereits erw"ahnte \textit{Industrie 4.0}, soll beispielsweise die industrielle Produktion mit moderner Informations- und Kommunikationstechnik intelligent verzahnt und digital vernetzt werden. Hierdurch soll die gesamte Wertsch"opfungskette eines Unternehmens optimiert und zeitgleich alle Phasen des Lebenszyklus eines Produktes mit eingebunden werden. Ebenfalls wird in Zukunft der Innovationstreiber des \textit{Internets der Dinge (IoT)} im industriellen Sektor zunehmend an Bedeutung gewinnen \cite{qdrei} (C. Lemke, W. Brenner). Hierbei stehen vor allem die weitere Automatisierung und Individualisierung von Entwicklungs- und Fertigungsprozessen im Vordergrund, sowie der zunehmende Anteil an IT-gebundenen Produkten innerhalb eines anderen physischen Produkts, den sogenannten \textit{\gls{embedded_systems}}.


\section{Motivation}
Trotz diverser positiver Ertr"age, welches das digitale Zeitalter mit sich bringt, ergeben sich aus diesem industriellen Wandel neue Herausforderungen, die es insbesondere f"ur Unternehmen zu bew"altigen gilt. Dieser Umbruch hat zufolge, dass sich Technologien im Laufe der Zeit permanent weiterentwickeln und somit auch deren Komplexit"at steigt. In der heutigen digitalen Zeit spielt ebenfalls eine schnelle und gute Reaktionszeit der Unternehmen 
eine entscheidende Rolle. 

Digitale Technologien sprengen die traditionellen Gesch"aftsmodelle in einer Art und Weise, dass alle Teilnehmer in der Wertsch"opfungskette ihre Rolle "uberdenken m"ussen. Traditionelle Teilnehmer, we beispielsweise die Zulieferer von Rohmaterial, Gro"sh"andler, Produzenten, Logistikunternehmen und Einzelh"andler werden obsolet oder ihre Rolle ver"andert sich signifikant. Wertsch"opfungsketten werden neu konfiguriert und folgen k"unftig neuen Regeln.

\section{Zukunftsf"ahig durch agile Transformation}
Inhalt ...



