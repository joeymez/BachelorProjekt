% !TEX root = Einleitung.tex

\chapter{Einleitung}
\minitoc
\vspace{1 cm} 

Die Vernetzung der Produktion in der Industrie durch die \gls{industrie_4.0}, die enorme Ansammlung und Auswertung von privaten und "offentlichen Daten (\gls{big_data}), sowie die Verkn"upfung von allt"aglichen Gegenst"anden mit dem Internet (\gls{iot}) -- dies sind nur einige Begrifflichkeiten, die das digitale Zeitalter definieren. Der Umbruch f"ur diese sogenannte \textit{Digitale Revolution} ist unter Fachkreisen auf den Beginn des 21. Jahrhunderts zur"uckzuf"uhren \([1]\). Seither hat diese neue digitale Welt sowohl im "offentlichen, im wirtschaftlichen, wie auch im privaten Sektor f"ur positive Bilanz und Begeisterung gesorgt. Nicht nur das es heutzutage beispielsweise mit Hilfe der Digitalisierung m"oglich ist Informationen schneller und einfacher zu verarbeiten, sondern das anhand dessen ebenfalls die Automatisierung in der Industrie vorangeschritten und kontinuierlich verbessert wird \([2]\). Auch au\ss{}erhalb der Wirtschaft wird von der heutigen digitalen Welt Gebrauch gemacht. Belege hierf"ur w"aren zum Beispiel das \textit{\gls{mobile_internet}}, das \textit{\gls{cloud_computing}}, oder
um einer der gr"o\ss{}ten Erfolge des digitalen Zeitalters zu nennen, das \textit{\gls{social_networking}}. 


\section{Motivation}
Inhalt ...

\section{Thesis - Agile Transformation im Konzern}
Inhalt ...



