\documentclass[12pt]{scrreprt}

%------------------------------------------------------
% LaTeX packages
%------------------------------------------------------
\usepackage[ngerman]{babel}
\usepackage{pdfpages}
\usepackage{geometry} 
\usepackage{url}
\usepackage{cite}
\usepackage[ngerman]{babel}          % Umlaute 
\usepackage[utf8]{inputenc}          % Eingabe Umlaute 
\usepackage{lmodern} 				 % neue deutsche Trennungsregeln, etc 
\usepackage{mathptmx} 				 % Schriftwart (Times New Roman)
\usepackage[T1]{fontenc}
\usepackage{etoolbox}
\usepackage[ngerman]{translator}
\usepackage[
nonumberlist, %keine Seitenzahlen anzeigen
acronym,      %ein Abkürzungsverzeichnis erstellen
toc,          %Einträge im Inhaltsverzeichnis
section]      %im Inhaltsverzeichnis auf section-Ebene erscheinen
{glossaries}


%--------------------------------------------------------------------------- 
% Überschriften 
%--------------------------------------------------------------------------- 
\setkomafont{disposition}{\sffamily} % Benutzt eine Schrift für alle Gliederungsebenen 
 
\addtokomafont{chapter}{\LARGE} % Formatierung Ebene 1 (https://www.latex-kurs.de/fragen/schriftgroesse.html)
\addtokomafont{section}{\Large} % Formatierung Ebene 2
\addtokomafont{subsection}{\large} % Formatierung Ebene 3


%------------------------------------------------------
% Settings
%------------------------------------------------------
\geometry{a4paper, left=30mm, right=25mm}


%------------------------------------------------------
% START OF DOCUMENT
%------------------------------------------------------
\begin{document}


%------------------------------------------------------
% Deckblatt
%------------------------------------------------------
\includepdf[pages={1}]{Deckblatt-BA.pdf}


%------------------------------------------------------
% Inhaltsverzeichniss
%------------------------------------------------------
\tableofcontents
\listoffigures
\listoftables


%------------------------------------------------------
% Dokument Anfang
%------------------------------------------------------
% !TEX root = Einleitung.tex

\chapter{Einleitung}


\section{Projektbegriff}


\section{Aufgaben des Projektmanagements}


\section{Projektmanagement Modelle}


\section{Problemstellung}
% !TEX root = Hauptteil.tex

\chapter{Hauptteil}


\section{Unternehmen MAN Truck Bus AG}


\section{Projekte mit agilen Ansätzen}


\section{Umsetzung}


\section{ ...}
% !TEX root = Resümee.tex

\chapter{Resümee}


\section{Zusammenfassung}


\section{Interpretation}


\section{Mögliche Forschungslücken (offene Fragen)}



%------------------------------------------------------
% Literatur
%------------------------------------------------------
\apptocmd{\thebibliography}{\csname phantomsection\endcsname\addcontentsline{toc}{chapter}{\bibname}}{}{}
\bibliography{literatur}{}
\bibliographystyle{plain}


%------------------------------------------------------
% END OF DOCUMENT
%------------------------------------------------------
\end{document}

